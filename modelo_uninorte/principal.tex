% ----------------------------------------------------------
% VERSÃO ORIGINAL
% ----------------------------------------------------------
% The Current Maintainer of this work is the abnTeX2 team, led
% by Lauro César Araujo. Further information are available on 
% http://abntex2.googlecode.com/

\documentclass[
% -- opções da classe memoir --
12pt,				% tamanho da fonte
openright,			% capítulos começam em pág ímpar (insere página vazia caso preciso)
oneside,			% para impressão em verso e anverso coloque twoside
a4paper,			% tamanho do papel. 
% -- opções da classe abntex2 --
%chapter=TITLE,		% títulos de capítulos convertidos em letras maiúsculas
%section=TITLE,		% títulos de seções convertidos em letras maiúsculas
%subsection=TITLE,	% títulos de subseções convertidos em letras maiúsculas
%subsubsection=TITLE,% títulos de subsubseções convertidos em letras maiúsculas
% -- opções do pacote babel --
english,			% idioma adicional para hifenização
french,				% idioma adicional para hifenização
spanish,			% idioma adicional para hifenização
brazil				% o último idioma é o principal do documento
]{abntex2}


% ----------------------------------------------------------
% PACOTES
% ----------------------------------------------------------
% ----------------------------------------------------------
% PACOTES BÁSICOS
% ----------------------------------------------------------
\usepackage{lmodern}            % Usa a fonte Latin Modern          
\usepackage[T1]{fontenc}        % Selecao de codigos de fonte.
\usepackage[utf8]{inputenc}     % Codificacao do documento (conversão automática dos acentos)
\usepackage{lastpage}           % Usado pela Ficha catalográfica
\usepackage{indentfirst}        % Indenta o primeiro parágrafo de cada seção.
\usepackage{color}              % Controle das cores
\usepackage{graphicx}           % Inclusão de gráficos
\usepackage{microtype}          % para melhorias de justificação
\usepackage{algorithm,algorithmic}          % Inserir código de linguagem de programação
\usepackage{float,array,multicol,multirow,booktabs}		% pacotes para ajuda em formatação de tabelas
\usepackage[none]{hyphenat} 	% Serve para eliminar a hifenização no documento, que é proibido segundo a noram NBR 14724/11 

        
% Pacotes de citações
\usepackage[brazilian,hyperpageref]{backref}     % Paginas com as citações na bibl
\usepackage[alf]{abntex2cite}   % Citações padrão ABNT

% CONFIGURAÇÕES DE PACOTES
% Configurações do pacote backref
% Usado sem a opção hyperpageref de backref
\renewcommand{\backrefpagesname}{Citado na(s) página(s):~}
% Texto padrão antes do número das páginas
\renewcommand{\backref}{}
% Define os textos da citação
\renewcommand*{\backrefalt}[4]{
    \ifcase #1 %
        Nenhuma citação no texto.%
    \or
        Citado na página #2.%
    \else
        Citado #1 vezes nas páginas #2.%
    \fi}%

% Pacotes adicionais, usados apenas no âmbito do Modelo Canônico do abnteX2
\usepackage{lipsum}             % para geração de dummy text

% TODO inserir seus pacotes aqui



% ESTILO
\usepackage{uninorte-abntex2}

% ----------------------------------------------------------
% CAPA E FOLHA DE ROSTO
% ----------------------------------------------------------
% ----------------------------------------------------------
% CAPA E FOLHA DE ROSTO
% ----------------------------------------------------------

% AUTOR
\newcommand{\tccautor}{Anderson Gadelha Fontoura}

\titulo{\uppercase{Modelo de TCC do Departamento de\\Engenharia Elétrica da UNINORTE}}
\autor{\tccautor}
\local{Manaus, AM}
\data{2016}
\orientador{Prof. Dr. Zé Ninguém}
\coorientador{Dr. Prof. Astucio Gilmar}
\instituicao{%
  Laureate International Universities
  \par
  Centro Universitário do Norte -- UNINORTE
  \par
  Departamento de Engenharia Elétrica}
%\tipotrabalho{Tese (Doutorado)}
\tipotrabalho{Trabalho de Conclusão de Curso (Monografia)}
% O preambulo deve conter o tipo do trabalho, o objetivo, 
% o nome da instituição e a área de concentração 
\preambulo{Trabalho de Conclusão de Curso submetido à Coordenação do Curso de Engenharia Elétrica do Centro Universitário do Norte (UNINORTE), como requisito parcial para obtenção do Título de Bacharel em Engenharia Elétrica.}



% ----------------------------------------------------------
% CONFIGURAÇÕES
% ----------------------------------------------------------

% Configurações de aparência do PDF final

% alterando o aspecto da cor azul
\definecolor{blue}{RGB}{41,5,195}

% informações do PDF
\makeatletter
\hypersetup{
	%pagebackref=true,
	pdftitle={\@title}, 
	pdfauthor={\@author},
	pdfsubject={\imprimirpreambulo},
	pdfcreator={LaTeX with abnTeX2},
	pdfkeywords={abnt}{latex}{abntex}{abntex2}{trabalho acadêmico}, 
	colorlinks=true,            % false: links em boxs; true: links coloridos
	linkcolor=blue,             % cor dos links de citação interna (\autoref)
	citecolor=blue,             % cor dos links de bibliografia
	filecolor=magenta,          % cor dos links dos arquivos
	urlcolor=blue,
	bookmarksdepth=4
}
\makeatother

% Espaçamentos entre linhas e parágrafos 
% O tamanho do parágrafo é dado por:
\setlength{\parindent}{1.25cm}

% Controle do espaçamento entre um parágrafo e outro:
\setlength{\parskip}{0.2cm}  % tente também \onelineskip

% compila o indice
\makeindex

% ----------------------------------------------------------
% INÍCIO DOCUMENTO
% ----------------------------------------------------------
\begin{document}
	
	% Retira espaço extra obsoleto entre as frases.
	\frenchspacing
	
	
	% Verifica hifenização (como as palavras devem ser separadas por - )
	\hyphenation{ARDUINO plan-ta es-ta-ção} 
	
	% ----------------------------------------------------------
	% ELEMENTOS PRÉ-TEXTUAIS
	% ----------------------------------------------------------
	% \pretextual
	
	% Capa
	\imprimircapa
	
	% Folha de rosto
	% (o * indica que haverá a ficha bibliográfica)
	\imprimirfolhaderosto*
	
	
	% Descomente apenas na versão final a ficha catalografica e a folha de aprovação
	% Não descomente a errata senão precisar
	%\input{elementos-pretextuais/ficha-catalografica}
	%\input{elementos-pretextuais/errata}
	% ----------------------------------------------------------
% INSERIR FOLHA DE APROVAÇÃO
% ----------------------------------------------------------
% Isto é um exemplo de Folha de aprovação, elemento obrigatório da NBR
% 14724/2011 (seção 4.2.1.3). Você pode utilizar este modelo até a aprovação
% do trabalho. Após isso, substitua todo o conteúdo deste arquivo por uma
% imagem da página assinada pela banca com o comando abaixo:
%
% \includepdf{folhadeaprovacao_final.pdf}
%
\begin{folhadeaprovacao}

  \begin{center}
    {\ABNTEXchapterfont\bfseries\large\imprimirautor}

    \vspace*{\fill}\vspace*{\fill}
    \begin{center}
      \ABNTEXchapterfont\bfseries\Large\imprimirtitulo
    \end{center}
    \vspace*{\fill}
    
    \hspace{.45\textwidth}
    \begin{minipage}{\textwidth}
        \imprimirpreambulo       
    \end{minipage}%
    %\vspace*{\fill}
    \vspace{1cm}
    Trabalho aprovado em: 14 de maio de 2016 \\
    \imprimirlocal
    \\
    \vspace{10mm}
    \uppercase{\textbf{banca examinadora}} 
   \end{center}     

   \assinatura{\textbf{\imprimirorientador \hspace{2mm}(Orientador)} \\ UNINORTE} 
   \assinatura{\textbf{Professor (Membro)} \\ UNINORTE}
   \assinatura{\textbf{Professor (Membro)} \\ UNINORTE}
   %\assinatura{\textbf{Professor} \\ Convidado 3}
   %\assinatura{\textbf{Professor} \\ Convidado 4}
      
%   \begin{center}
%    \vspace*{0.5cm}
%    {\large\imprimirlocal}
%    \par
%    {\large\imprimirdata}
%    \vspace*{1cm}
%  \end{center}
  
\end{folhadeaprovacao}


	\input{elementos-pretextuais/dedicatoria}
	\input{elementos-pretextuais/agradecimentos}
	% ----------------------------------------------------------
% EPÍGRAFE
% ----------------------------------------------------------
\begin{epigrafe}
    \vspace*{\fill}
	\begin{flushright}
		\begin{minipage}{.5\textwidth}
			\textit{``Não vos amoldeis às estruturas deste mundo, 
			mas transformai-vos pela renovação da mente, 
			a fim de distinguir qual é a vontade de Deus: 
			o que é bom, o que Lhe é agradável, o que é perfeito''.}
			\begin{flushright}
				(Bíblia Sagrada, Romanos 12, 2)
			\end{flushright}
		\end{minipage}
	\end{flushright}
\end{epigrafe}


	\input{elementos-pretextuais/resumos}
	
	% ----------------------------------------------------------
	% inserir lista de ilustrações
	% ----------------------------------------------------------
	\pdfbookmark[0]{\listfigurename}{lof}
	\listoffigures*
	\cleardoublepage
	
	% ----------------------------------------------------------
	% inserir lista de tabelas
	% ----------------------------------------------------------
	\pdfbookmark[0]{\listtablename}{lot}
	\listoftables*
	\cleardoublepage
	
	% ----------------------------------------------------------
	% inserir lista siglas e abreviaturas
	% ----------------------------------------------------------
	\input{elementos-pretextuais/siglas}
	
	% ----------------------------------------------------------
	% inserir lista símbolos
	\input{elementos-pretextuais/simbolos}
	% ----------------------------------------------------------
	
	% ----------------------------------------------------------
	% inserir o sumario
	% ----------------------------------------------------------
	\pdfbookmark[0]{\contentsname}{toc}
	\tableofcontents*
	\cleardoublepage
	
	% ----------------------------------------------------------
	% ELEMENTOS TEXTUAIS
	% ----------------------------------------------------------
	\textual
	
	% ----------------------------------------------------------
	% Introdução (exemplo de capítulo sem numeração, mas presente no Sumário)
	% ----------------------------------------------------------
	% TODO inserir seu capítulo 1 aqui
	\chapter*[Introdução]{Introdução}
	\addcontentsline{toc}{chapter}{Introdução}
	% ----------------------------------------------------------
	\include{elementos-textuais/introducao}
	
	% ----------------------------------------------------------
	% Capitulo com exemplos de comandos inseridos de arquivo externo
	% Este capitulo deve conter a contextualização, justificativa, problemática, hipoteses e objetivos de seu trabalho 
	% ----------------------------------------------------------
	\include{elementos-textuais/capitulo-1}
	
	% ----------------------------------------------------------
	% Capitulo com exemplos de comandos inseridos de arquivo externo
	% Ess capitulo deve conter a fundamentação téorica de seu trabalho 
	% ----------------------------------------------------------
	\include{elementos-textuais/capitulo-2}
	% ----------------------------------------------------------
	
	% ----------------------------------------------------------
	% Capitulo com exemplos de comandos inseridos de arquivo externo
	% Ess capitulo deve conter os trabalhos correlatos ao seu trabalho 
	% ----------------------------------------------------------
	%\include{elementos-textuais/capitulo-3}
	% ----------------------------------------------------------
	
	% ----------------------------------------------------------
	% Capitulo com exemplos de comandos inseridos de arquivo externo
	% Ess capitulo deve conter método proposto de seu trabalho 
	% ----------------------------------------------------------
	%\include{elementos-textuais/capitulo-4}
	% ----------------------------------------------------------
	
	% ----------------------------------------------------------
	% Capitulo com exemplos de comandos inseridos de arquivo externo
	% Ess capitulo deve conter os resultados parciais e finais de seu trabalho 
	% ----------------------------------------------------------
	%\include{elementos-textuais/capitulo-5}
	% ----------------------------------------------------------
	
	% Se precisar, insira mais capitulos...
	
	% Finaliza a parte no bookmark do PDF
	% para que se inicie o bookmark na raiz
	% e adiciona espaço de parte no Sumário
	% ----------------------------------------------------------
	\phantompart
	
	% ----------------------------------------------------------
	% Conclusão (outro exemplo de capítulo sem numeração e presente no sumário)
	% ----------------------------------------------------------
	\chapter*[Conclusão]{Conclusão}
	\addcontentsline{toc}{chapter}{Conclusão}
	% ----------------------------------------------------------
	\input{elementos-textuais/conclusao}
	
	% ----------------------------------------------------------
	% ELEMENTOS PÓS-TEXTUAIS
	% ----------------------------------------------------------
	\postextual
	% ----------------------------------------------------------
	
	% ----------------------------------------------------------
	% Referências bibliográficas
	% ----------------------------------------------------------
	\bibliography{elementos-postextuais/referencias}
	
	% ----------------------------------------------------------
	% Glossário
	% ----------------------------------------------------------
	%
	% Consulte o manual da classe abntex2 para orientações sobre o glossário.
	%
	%\glossary
	
	% ----------------------------------------------------------
	% Apêndices
	% ----------------------------------------------------------
	\input{elementos-postextuais/apendices}	
	
	% ----------------------------------------------------------
	% Anexos
	% ----------------------------------------------------------
	\input{elementos-postextuais/anexos}
	
	%---------------------------------------------------------------------
	% INDICE REMISSIVO
	%---------------------------------------------------------------------
	\phantompart
	\printindex
	%---------------------------------------------------------------------
	
\end{document}